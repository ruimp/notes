\documentclass{article}

\usepackage[a4paper, margin=2cm]{geometry}
\usepackage[utf8]{inputenc}
\usepackage[english]{babel}
\usepackage{amsmath}
\usepackage{amssymb}
\usepackage{amsthm}
\usepackage{tensor}
\usepackage{microtype}
\usepackage{booktabs}
\usepackage{marginnote}
\usepackage{tikz}
\usepackage{hyperref}
\usepackage{dsfont}
\usetikzlibrary{cd}
\usepackage{wrapfig}
\usepackage[T1]{fontenc}
\usepackage{newpxtext,eulerpx}


\newtheorem{theorem}{Theorem}
\newtheorem{lemma}{Lemma}
\newtheorem{corollary}{Corollary}[theorem]
\theoremstyle{definition}
\newtheorem{definition}{Definition}
\newtheorem{proposition}{Proposition}
\newtheorem*{remark}{Remark}
\newtheorem{example}{Example}

\title{Notes on Superanalysis}
\author{Rui Peixoto}
\date{}

\begin{document}
\maketitle

\section{Grassmann Algebra}

\subsection{Associative Algebras}

\subsubsection{Algebras}

\begin{definition}
    An \emph{algebra} is a vector space $\mathfrak{A}$ over a field $\mathbb{K}$ equiped with a product satisfying $\forall u, v, w \in \mathfrak{A}$ and $\forall \alpha \in \mathbb{K}$
    \begin{itemize}
        \item $u(v + w) = uv + uw$
        \item $(u + v)w = uw + vw$
        \item $\alpha (u v) = (\alpha u) v = u (\alpha v)$.
    \end{itemize}

    The algebra $\mathfrak{A}$ is said to be \emph{associative} if $\forall u, v, w \in \mathfrak{A}$
    \begin{equation*}
        (uv)w = u(vw)
    \end{equation*}
    and \emph{unitary} (or algebra with a unit) if $\exists e \in \mathfrak{A}$ such that $\forall u \in \mathfrak{A}$
    \begin{equation*}
        eu = ue = u.
    \end{equation*}
    Furthermore, $\mathfrak{A}$ is \emph{commutative} if $\forall u, v \in \mathfrak{A}$
    \begin{equation*}
        uv = vu
    \end{equation*}
    and $\dim \mathfrak{A} = n$ for $0 \leq n \leq \infty$ if $\mathfrak{A}$ has dimension $n$ as a vector space.
\end{definition}

\begin{remark}
    We will consider only the cases $\mathbb{K} \in \{\mathbb{R}, \mathbb{C}\}$.
\end{remark}

\begin{example}
    There exist important algebras without a unit. Important examples of such algebras are \emph{Lie algebras}.
\end{example}

\begin{proposition}
    Let $\mathfrak{A}$ be a non-trivial Lie algebra. Then $\mathfrak{A}$ does not have a unit.
\end{proposition}
\begin{proof}
    Let $x, y \in \mathfrak{A}$. Assume $\mathfrak{A}$ admits a unit $e \in \mathfrak{A}$, that is $\left[x, e \right] = x = \left[ e, x \right]$ for all $x \in \mathfrak{A}$. By the anticommutativity of the bracket this implies that
    \begin{equation*}
        e = \left[ e, e \right] = -\left[ e, e \right]
        \Longrightarrow e = 0
    \end{equation*}
    which contradicts the fact that $\mathfrak{A}$ is non-trivial.
\end{proof}

\begin{proposition}
    Let $M$ be a set and $\mathfrak{A}$ an algebra over a field $\mathbb{K}$. Then $\mathfrak{A}^M = \left\{ f: M \longrightarrow \mathfrak{A} \right\}$ is an algebra over $\mathbb{K}$. If $\mathfrak{A}$ is associative or commutative, then so is $\mathfrak{A}^M$.
\end{proposition}
\begin{proof}
    Let $f, g, h \in \mathfrak{A}^M$, and $\alpha \in \mathbb{K}$. Then $\forall x \in M$
    \begin{align*}
        f(x) \left[ g(x) + h(x) \right] = f(x) g(x) + f(x)h(x) 
        & \Longrightarrow f(g + h) = fg + fh \\
        \left[f(x) + g(x) \right] h(x)  = f(x) h(x) + g(x)h(x) 
        & \Longrightarrow (f + g)h = fh + gh \\
        \left[ \alpha f(x) \right] g(x) = f(x) \left[ \alpha g(x) \right]
         = \alpha f(x) g(x) & \Longrightarrow (\alpha f) g = f(\alpha g) = \alpha fg
    \end{align*}
    so $\mathfrak{A}^M$ is an algebra over $\mathbb{K}$. Associativity and commutativity inherited from $\mathfrak{A}$ are shown analogously.
\end{proof}

\begin{example}
    Consider a domain $U \subseteq \mathbb{R}^p$. Then $
        \mathcal{A}(U) = \left\{ f \in \mathbb{K}^U \ \middle| \ f \text{ is smooth} \right\} $
    is an algebra over $\mathbb{K}$. $\mathcal{A}(U)$ admits a subalgebra $\mathcal{P}(U) = \left\{ f \in \mathcal{K}^U \ \middle| \ f \text{ is polynomial} \right\}$.
\end{example}

\subsubsection{Generators}

\begin{definition}
    Let $\Sigma \subseteq \mathfrak{A}$, then the set of polynomials on the elements $a_i \in \Sigma$
    \begin{equation}
        \label{eq:generators_algebra}
        \mathfrak{A}(\Sigma) = f_0 \sum_{k \geq 1} \sum_{i_1 \dots i_k}
        f_{i_1\dots i_k} a_{i_1} \dots a_{i_k}
    \end{equation}
    for $f_{i_1 \dots i_k} \in \mathbb{K}$, where the sum is finite, is the \emph{subalgebra generated by} $\Sigma$. If $\mathfrak{A} = \mathfrak{A}(\Sigma)$ the set $\Sigma$ is said to be a \emph{system of algebraic generators} of $\mathfrak{A}$. For $\mathfrak{A}$ a topological algebra, the set $\Sigma$ is called a \emph{system of topological generators} of $\mathfrak{A}$ if $\mathfrak{A} = \overline{\mathfrak{A}(\Sigma)}$.
\end{definition}

\begin{proposition}
    Let $\mathfrak{A}$ be a finite-dimensional topological algebra. Any system of topological generators of $\mathfrak{A}$ admits a finite subset constituting a also a system of algebraic generators. This is not necessarily true for infinite-dimensional algebras.
\end{proposition}
\begin{proof}
    Page 31-32 of \cite{berezin_introduction_1987}.
\end{proof}

\begin{definition}
    Let $\mathfrak{R}$ be a left (right) ideal of an algebra $\mathfrak{A}$. A subset $\Sigma \subseteq \mathfrak{R}$ is a \emph{system of generators} of $\mathfrak{R}$ or \emph{generating set} if $\forall x \in \mathfrak{R}$ can be writen as a polynomial on elements $\xi_i \in \Sigma$
    \begin{equation}
        \label{eq:generators_ideals}
        x = \sum_{k \geq 1} \sum_{i_1 \dots i_k} a_{i_1 \dots i_k} \xi_1 \dots \xi_k
    \end{equation}
    for $a_{i_1 \dots i_k} \in \mathfrak{A}$, where the sum is finite. Elements of the form \eqref{eq:generators_ideals} are called left (right) polynomials. If $\mathfrak{A}$ is a topological algebra, $\Sigma$ is a \emph{system of topological generators} if the collection of left polynomials of elements in $\Sigma$ with coefficients in $\mathfrak{A}$ are dense in $\mathfrak{R}$.
\end{definition}


%%%%%%%%%%%%%%%%%%%%%%%%%%%%%%%%%%%%%%%


\subsection{Grassmann Algebras}
\subsubsection{Properties of Grassmann Algebras}


\begin{definition}
    An associative algebra $\Lambda$ with unit is called a \emph{Grassmann} or \emph{exterior} algebra if it contains a system of generators $\{ \xi_i \}_{i \in I}$ such that $\forall i, k \in I$
    \begin{equation}
        \label{def:grassmann_commutation}
        \xi_i \xi_k + \xi_k \xi_i = 0.
    \end{equation}
    The generators $\xi_i$ are called \emph{canonical}. To make the choice of system of generators explicit one may write $\Lambda = \Lambda (\xi_i, \dots, \xi_q)$.
\end{definition}

\begin{remark}
    The commutation relations in \eqref{def:grassmann_commutation} imply that any element $f \in \Lambda(\xi_1, \dots, \xi_q)$ can be writen as
    \begin{equation}
        \label{eq:grassmann_element}
        f = f(\xi) = \sum_{k \geq 0} \sum_{i_1 \dots i_k}
        f_{i_1 \dots i_k} \xi_{i_1} \dots \xi_{i_k}.
    \end{equation}
    The term with $k = 0$ is proportional to the unit in $\Lambda$. This decomposition is unique if one requires $f_{i_1 \dots i_k}$ to be skew-symmetric with respect to $i_1, \dots, i_k$.
\end{remark}

\begin{definition}
    Let $f \in \Lambda(\xi_1, \dots, \xi_q)$. Let $n \in \mathbb{N}$ be such that $f_{i_1 \dots i_k} = 0$ for $k < n$ and there exists some non zero coefficient for $k = n$. Then $n$ is called the \emph{degree} of $f$, writen as $\deg f = n$. If the nonvanishing coefficients of $f$ appear only for $k = n$, then $f$ is said to be \emph{homogeneous of degree $n$ with respect to $\deg$}.
\end{definition}

\begin{proposition}
    The degree of $f$ does not depend on the choice of system of generators of $\Lambda$. Wether $f$ is homogeous with respect to $\deg$ does, in general, depend on the choice of system of generators.
\end{proposition}
\begin{proof}
    Pages 36-37 of \cite{berezin_introduction_1987}.
\end{proof}

\begin{definition}
    Let $f \in \Lambda$. Then 
    \begin{equation*}
        \Lambda^{(k)} = \left\{ f \in \Lambda \ \middle| \ \deg f \geq k \right\}
    \end{equation*}
    is and ideal of $\Lambda$ for $k \in \mathbb{N}$. These are ordered by inclusion
    \begin{equation*}
        \Lambda = \Lambda^{(0)} \supseteq \Lambda^{(1)} \supseteq \dots
        \supseteq \Lambda^{(k)} \supseteq \dots.
    \end{equation*}
\end{definition}

\begin{definition}
    Let $f \in \Lambda(\xi_1, \dots, \xi_q)$. If the decomposition of $f$ has in \eqref{eq:grassmann_element} is nonzero only for even (odd) values of $k$, then $f$ is said to be \emph{even (odd) with respect to the system of generators} $\left\{ \xi_1, \dots, \xi_q \right\}$. Even and odd elements are \emph{homogeneous with respect to parity}.
\end{definition}

\begin{remark}
    Seperating the sum in \eqref{eq:grassmann_element} into even and odd parts, it is becomes clear that
    \begin{equation*}
        \Lambda(\xi_1, \dots, \xi_q) = \tensor[^0]{\Lambda(\xi_1, \dots, \xi_q)}{} \oplus \tensor[^1]{\Lambda(\xi_1, \dots, \xi_q)}{}.
    \end{equation*}
\end{remark}

\begin{remark}
    The even subspace $\tensor[^0]{\Lambda(\xi_1, \dots, \xi_q)}{} \subseteq \Lambda$ contains the unit. In fact, it forms a subalgebra of $\Lambda$, so one can write $\tensor[^0]{\Lambda}{} \subseteq \Lambda$ with no ambiguity arising from the choice of system of generators.
\end{remark}

\begin{definition}
    Let $\Lambda(\xi_1, \dots, \xi_q)$ be a Grassmann algebra. The even and odd elements form linear subspaces $\tensor[^0]{\Lambda(\xi_1, \dots, \xi_q)}{}$ and $\tensor[^1]{\Lambda(\xi_1, \dots, \xi_q)}{}$, respectively. We define the \emph{parity} of $f \in \Lambda(\xi_1, \dots, \xi_q)$ by
    \begin{equation*}
        \alpha(f) = \left\{
        \begin{array}{ll}
            0 & \text{if } f \in \tensor[^0]{\Lambda(\xi_1, \dots, \xi_q)}{} \\
            1 & \text{if } f \in \tensor[^1]{\Lambda(\xi_1, \dots, \xi_q)}{}
        \end{array}\right. \ .
    \end{equation*}
    Then, for $f, g \in \Lambda(\xi_1, \dots, \xi_q)$ we have
    \begin{equation*}
        \alpha(fg) = \alpha(f) + \alpha(g) \mod 2
    \end{equation*}
    and
    \begin{equation}
        \label{eq:grassmann_commutation}
        fg = (-1)^{\alpha(f) \alpha(g)}gf.
    \end{equation}

    \begin{proposition}
        The subspace $\tensor[^0]{\Lambda}{} \subseteq \Lambda$ does not depend on the choice of system of generators.
    \end{proposition}
    \begin{proof}
        We will show that
        \begin{equation*}
            \tensor[^0]{\Lambda}{} =
            \left\{ f \in \Lambda \ \middle| \ fg = gf, \forall g \in \Lambda \right\}.
        \end{equation*}
        Let $f \in \tensor[^0]{\Lambda}{}$. Then by \eqref{eq:grassmann_commutation} it follows that $fg = gf$ for all $g \in \Lambda$. On the other hand, let $f \in \Lambda$ commute with every element $g \in \Lambda$. In particular, we might pick $g$ odd, so $\alpha(g) = 1$. From \eqref{eq:grassmann_commutation} it follows that $\alpha(f) \alpha(g) = 0$, thus $\alpha(f) = 0$.
    \end{proof}

    \begin{proposition}
        \label{prop:properties_odd_elements}
        Let $f, g \in \tensor[^1]{\Lambda(\xi_1, \dots, \xi_q)}{}$ and $h \in \tensor[^0]{\Lambda}{}$. Then
        \begin{itemize}
            \item $fg = -gf \in \tensor[^0]{\Lambda}{}$
            \item $fh \in \tensor[^1]{\Lambda(\xi_1, \dots, \xi_q)}{}$.
        \end{itemize}
    \end{proposition}
    \begin{proof}
        These properties of odd elements follow directly from \eqref{eq:grassmann_element} and \eqref{eq:grassmann_commutation}.
    \end{proof}

\end{definition}

\subsubsection{Grassmann Algebra Generator Systems}

\begin{theorem}
    \label{thm:grassmann_generators}
    Any system of generators of a Grassman algebra $\Lambda_q$ contains a subsystem consisting of $q$ generators, and there exists no subsystem with fewer generators. Furthermore, let $\left\{ \xi_i \right\} \subseteq \Lambda_q$ be a subset of nilpotent elements whose union with the unit forms a system of generators of $\Lambda_q$. Then a subset $\left\{ \eta_j \right\} \subseteq \Lambda_q$ has the same property if and only if
    \begin{equation*}
        \eta_i = \sum_{j = 1}^{q} a_{ij} \xi_j + \sigma_j
    \end{equation*}
    for $\deg \sigma_i > 1$ and $\det \left\| a_{ij} \right\| \neq 0$.
\end{theorem}
\begin{proof}
    Pages 39-41 of \cite{berezin_introduction_1987}.
\end{proof}

\begin{theorem}
    \label{thm:grassmann_canonical_generators}
    Let $\left\{ \xi_i \right\}$ be a system of canonnical generators of the Grassmann algebra $\Lambda(\xi_1, \dots, \xi_q)$. Then
    \begin{enumerate}
        \item a collection of off elements $\left\{ \eta_j \right\}$ with respect to $\left\{ \xi_i \right\}$ together with the unit constitutes a system of canonical generators of $\Lambda_q$ if and only if
        \begin{equation*}
            \eta_i = \sum_{j = 1}^q \psi_{ij} \xi_j + \alpha_i, \quad \forall i \in \left\{ 1, \dots, q \right\}
        \end{equation*}
        with $\det \left\| \psi_{ij} \right\| \neq 0$ and $\deg \alpha_i \geq 3$.
        \item an arbitrary collection $\left\{ \eta_i \right\}$ together with $1$ forms a system of canonical generators of $\Lambda_q$ if and only if
        \begin{equation}
            \label{eq:canonical_generators}
            \eta_i = \zeta_i(1 + \mu), \quad \forall i \in \left\{ 1, \dots, q \right\}
        \end{equation}
        where $\left\{ \zeta_i \right\}$ is a system of odd (with respect to $\left\{ \xi_i \right\}$) canonical generators of $\Lambda_q$ and $\mu$ is also odd.
    \end{enumerate}
\end{theorem}
\begin{proof}
    The first item follows from theorem \ref{thm:grassmann_generators}. Consider then the second item. Suppose that $\left\{ \eta_i \right\}$ are of the form \eqref{eq:canonical_generators}. Then
    \begin{equation*}
        \eta_i \eta_j = \zeta_i (1 + \mu) \zeta_j (1 + \mu)
        = \zeta_i \zeta_j ( 1 - \mu) (1 + \mu)
        = \zeta_i \zeta_j (1 + \mu^2) = \zeta_i \zeta_j
    \end{equation*}
    Then $\forall f \in \tensor[^1]{\Lambda}{_q}$ we have that $f(\eta) = f(\zeta)(1 + \mu(\zeta))$ by substituting in \eqref{eq:grassmann_element}. In particular
    \begin{equation*}
        \mu(\eta) = \mu(\zeta)(1 + \mu(\zeta)) = \mu(\zeta) + \mu(\zeta)^2 = \mu(\zeta)
    \end{equation*}
    therefore $\xi_i = \eta_i(1 + \mu(\eta))^{-1} = \eta_i (1 - \mu(\eta))$, so $\left\{ \eta_i \right\}$ is a system of canonical generators. Now assume $\left\{ \eta_i \right\}$ is a system of canonical generators. Then we may write $\eta_i = \zeta_i + \vartheta_i$ for $\zeta_i \in \tensor[^1]{\Lambda}{_q}$ and $\vartheta_i \in \tensor[^0]{\Lambda}{_q}$ with $\deg \vartheta_i > 1$, for all $i \in \left\{ 1, \dots, q \right\}$. From theorem \ref{thm:grassmann_generators} follows that $\left\{ \zeta_i \right\}$ form a system of odd generators, thus canonical. Hence
    \begin{equation*}
        \eta_i \eta_j + \eta_j \eta_i = (\zeta_i + \vartheta_i)(\zeta_j + \vartheta_j) + (\zeta_j + \vartheta_j)(\zeta_i + \vartheta_i)
        = 2 (\zeta_i \vartheta_j + \zeta_j \vartheta_i) = 0
    \end{equation*}
    for all $i, j \in \left\{ 1, \dots, q \right\}$. Setting $i = j$
    \begin{equation*}
        \eta_i^2 = (\zeta_i + \vartheta_i)^2 = \zeta_i \vartheta_i + \vartheta_i \zeta_i = 0
    \end{equation*}
    so $\vartheta_i = \zeta_i \mu_i$ for some $\mu_i \in \tensor[^1]{\Lambda}{_q}$. Suppose that there exits $\mu \in \tensor[^1]{\Lambda}{_q}$ such that $\mu = \mu_i + \zeta_i \lambda_i$ for some $\lambda_i \in \tensor[^1]{\Lambda}{_q}$ and $i \in \left\{ 1, \dots, q \right\}$. Then
    \begin{equation*}
        \eta_i = \zeta_i + \vartheta_i = \zeta_i (1 + \mu_i)
        = \zeta_i (1 + \mu_i + \zeta_i \lambda_i) = \zeta_i (1 + \mu).
    \end{equation*}
    As such, lemma \eqref{lemma:1} conclude the proof.
\end{proof}

\begin{lemma}
    \label{lemma:1}
    There exists $\mu \in \tensor[^1]{\Lambda}{_q}$ such that $\mu = \mu_i + \zeta_i \lambda_i$ for all $i \in \left\{ 1, \dots, q \right\}$.
\end{lemma}
\begin{proof}
    Noticing that for all $i, j \in \left\{ 1, \dots, q \right\}$
    \begin{equation*}
        0 = \zeta_i \zeta_j (\mu - \mu)
        = \zeta_i \zeta_j (\mu_i - \mu_j + \zeta_i \lambda_i - \zeta_j \lambda_j)
        = \zeta_i \zeta_j (\mu_i - \mu_j)
    \end{equation*}
    implies $\mu_i - \mu_j = \zeta_i \lambda + \zeta_j \kappa$ for $\lambda, \kappa \in \Lambda_q$. The proof follows by induction as in page 42 of \cite{berezin_introduction_1987}.
\end{proof}

\begin{corollary}
    Let $\left\{ \xi_i \right\}$ be a system of canonical generators of $\Lambda$ and $\mu \in \tensor[^1]{\Lambda(\xi_1, \dots, \xi_q)}{}$. Then
    \begin{equation*}
        \tensor[^{1, \mu}]{\Lambda}{}
        = \left\{ f(1 + \mu) \ \middle| \ f \in \tensor[^1]{\Lambda(\xi_1, \dots, \xi_q)}{} \right\}.
    \end{equation*}
    is the space of odd elements with respect to some system of canonica generators of $\Lambda$.
\end{corollary}


\subsubsection{Parity Automorphisms}


\begin{definition}
    Let $\Lambda$ be a Grassmann algebra. Let $\tensor[^0]{\Lambda}{}$ and $\tensor[^1]{\Lambda}{}$ be its even and odd componenets with respect to some system of generators. A linear transformation $A: \Lambda \rightarrow \Lambda$ is a \emph{parity automorphism} if
    \begin{enumerate}
        \item $A^2 = \mathds{1}$, the identity automorphism
        \item $A f = f \Leftrightarrow f \in \tensor[^0]{\Lambda}{}$.
    \end{enumerate}
\end{definition}

\begin{theorem}
    \label{thm:parity_aut_generators}
    Let $\Lambda$ be a Grassmann algebra and $A$ a parity automophism of $\Lambda$. The space of eigenvectors of $A$ with eigenvalue $-1$ coincides with the space of odd elements with respect to a system of canonical generators.
\end{theorem}
\begin{proof}
    Page 43 of \cite{berezin_introduction_1987}.
\end{proof}

\begin{corollary}
    Let $\Lambda$ be a Grassmann algebra. A subspace $\tensor[^1]{\Lambda}{} \subseteq \Lambda$ such that $\Lambda = \tensor[^0]{\Lambda}{} \oplus \tensor[^1]{\Lambda}{}$ satisfying properties of proposition \ref{prop:properties_odd_elements} coincides with the space of odd elements with respect to a system of canonical generators.
\end{corollary}
\begin{proof}
    Define $A: \Lambda \rightarrow \Lambda$ by
    \begin{equation*}
        Af = \left\{
        \begin{array}{cr}
            f & \text{if } f \in \tensor[^0]{\Lambda}{} \\
            -f & \text{if } f \in \tensor[^1]{\Lambda}{}
        \end{array} \right.
    \end{equation*}
    which is a parity automorphism. The corollary follows from theorem \ref{thm:parity_aut_generators}.
\end{proof}


\subsubsection{Automorphisms of Grassmann Algebras}

\begin{proposition}
    Theorem \ref{thm:grassmann_generators} gives a discription of all automophisms of Grassmann algebras.
\end{proposition}
\begin{proof}
    Let $T: \Lambda \rightarrow \Lambda$ be such an automorphism and $\left\{ \xi_i \right\}$ a system of canonical generators of $\Lambda$. Then $T$ maps $\left\{ \xi_i \right\}$ into a system of canonical generators. On the other hand, given another system of canonical generators $\left\{ \eta_i \right\}$, then one may contruct $T$ such that $T \xi_i = \eta_i$ for all $i \in \left\{ 1, \dots, q \right\}$, and this uniquely defines $T$.
\end{proof}

\begin{definition}
    Let $\Lambda$ be a Grassmann algebra. The group of automorphisms of $\Lambda$ is denoted by $\text{Aut} (\Lambda)$. Let $A \in \text{Aut}(\Lambda)$ be a parity automorphism. Then one writes $\tensor[^{A, 1}]{\Lambda}{} \subseteq \Lambda$ to denote the subspace consisting of eigenvectors of $A$ corresponding to eigenvector $-1$. By theorem \ref{thm:parity_aut_generators} follows that $f \in \tensor[^{A, 1}]{\Lambda}{}$ are odd with respect to some system of canonical generators.
\end{definition}

\begin{definition}
    Given a Grassmann algebra $\Lambda$ and a parity automorphism $A$, we define $\text{Aut}_A^1(\Lambda) \subseteq \text{Aut}(\Lambda)$ to be the subgroup of automorphisms that preserve $\tensor[^{A, 1}]{\Lambda}{}$. Let $\left\{ \xi_i \right\} \subseteq \tensor[^{A, 1}]{\Lambda}{}$ be a system of canonical generators of $\Lambda$. Then we denote by $\text{Aut}_A^1(\Lambda)$ the subgroup of automorphisms of $\Lambda$ of the form
    \begin{equation*}
        T \xi_i = \xi_i (1 + \mu(\xi))
    \end{equation*}
    for $\mu(\xi) \in \tensor[^{A, 1}]{\Lambda}{}$.
\end{definition}

\begin{proposition}
    Let $\Lambda$ be a Grassmann algebra. The following properties hold:
    \begin{enumerate}
        \item Let $T \in \text{Aut}(\Lambda)$. Then $T$ can be represented uniquely as $T = T_1 T_2$ for $T_1 \in \text{Aut}_A^1(\Lambda)$ and $T_2 \in \text{Aut}_A^0(\Lambda)$.
        \item Let $T \in \text{Aut}_A^1(\Lambda)$, then $T$ acts trivially on even elements.
        \item Let $T \in \text{Aut}_A^1(\Lambda)$, then $T$ acts on odd elements $f \in \Lambda$ with respect to a system of canonical generators $\left\{ \xi_i \right\}$ according to
        \begin{equation*}
            (Tf)(\xi) = f(\xi)(1 + \mu(\xi)).
        \end{equation*}
        \item $\text{Aut}_A^1(\Lambda)$ is commutative and isomorphic to $\left( \tensor[^{A, 1}]{\Lambda}{}, + \right)$ as groups.
        \item $\text{Aut}_A^1(\Lambda)$ is a normal subgroup of $\text{Aut}(\Lambda)$.
        \item Let $A$ and $B$ be parity automorphisms of $\Lambda$. Then the subgroups $\text{Aut}_A^0(\Lambda)$ and $\text{Aut}_B^0(\Lambda)$ (or $\text{Aut}_A^1(\Lambda)$ and $\text{Aut}_B^1(\Lambda)$) are conjugate.
        \item The set of elements $T \in \text{Aut}(\Lambda)$ of the form
        \begin{equation*}
            T \xi_i = \xi_i + u(\xi)
        \end{equation*}
        with $\deg u \geq n$ form a normal subgroup $\text{Aut}_n(\Lambda) \subseteq \text{Aut}(\Lambda)$.
    \end{enumerate}
\end{proposition}
\begin{proof}
    Let $\Lambda(\xi_1, \dots, \xi_q)$ be a Grassmann algebra and $A$ a parity automophism of $\Lambda$.
    \begin{enumerate}
        \item Follows from the decomposition $\Lambda = \tensor[^0]{\Lambda}{} \oplus \tensor[^1]{\Lambda}{}$.
        \item Notice that
        \begin{equation*}
            T(\xi_1 \xi_2) = \xi_1(1 + \mu)\xi_2(1 + \mu) = \xi_1 \xi_2 (1 - \mu)(1 + \mu) = \xi_1 \xi_2.
        \end{equation*}
        Writing any element $f \in \tensor[^0]{\Lambda}{}$ as in \eqref{eq:grassmann_element}, the result follows.
        \item Follows from \eqref{eq:grassmann_element}, analogously to above.
        \item Let $T_1, T_2 \in \text{Aut}_A^1(\Lambda)$. Then
        \begin{equation*}
            T_1 T_2 \xi_i = T_1 \xi_i (1 + \mu_2(\xi))
            = \xi_i(1 + \mu_2(\xi))(1 + \mu_1(\xi))
            = \xi_i \left[1 + \mu_2(\xi) + \mu_1(\xi) + \mu_1(\xi) \mu_2(\xi) \right].
        \end{equation*}
        But theorem \ref{thm:grassmann_canonical_generators} states that $\eta_i = T_1 T_2 \xi_i$ must of the form
        \begin{equation*}
            \eta_i = \zeta_i ( 1 + \mu), \quad \zeta_i = \sum_{j = 1}^q a_{ij} \xi_j + \sigma_i
        \end{equation*}
        with $\deg \sigma_i \geq 3$ and $\mu, \sigma_i \in \t,ensor[^1]{\Lambda(\xi_1, \dots, \xi_q)}{}$, thus $\mu_1(\xi) \mu_2(\xi) = 0$.
    \end{enumerate}
\end{proof}







\newpage
%%%%%%%%%%%%%%%%%%%%%%%%%%%%%%%%%%%%%%%

\section{Supermanifolds}

\subsection{Sheaves}

\begin{definition}
    Let X be a topological space. We define Op$(X)$ to be the \emph{category of open sets of X}. The objects of Op$(X)$ are open subsets $U \subseteq X$ and morphisms are inclusions of open sets $V \subseteq U \subseteq X$ into eachother.

    \begin{figure}[ht]
        \centering
        \begin{tikzcd}[column sep=small]
            V \arrow[rr, hook] \arrow[dr] &  & U \arrow[dl] \\
            & X &
        \end{tikzcd}
        \caption{Morphisms in Op$(X)$.}
    \end{figure}
\end{definition}

\begin{remark}
    Morphisms in Op$(X)$ induce a partial ordering of Op$(X)$.
\end{remark}

\begin{definition}
    Let $X$ be a topological space and $C$ a locally small category\footnote{Consider categories such as Set, Ab, Vect, or Ring.}. A \emph{presheaf} over $X$ with values in $C$ is a contravariant functor $F:\text{Op}(X) \rightarrow C$.

A presheaf includes
\begin{enumerate}
    \item an assignment of an object $F(U)$ from the category $C$ to each open set $V \subseteq X$,
    \item an assignment of a morphism $\rho_V^U: F(U) \rightarrow F(V)$ from the category $C$ to each pair $V \subseteq U$ in Op$(X)$,
    \item an identity morphism $\rho_U^U = \text{id}$,
    \item $\rho_W^V \rho_V^U= \rho_W^U$ for open sets $W \subseteq V \subseteq U \subseteq X$.
\end{enumerate}

\begin{figure}[h]
    \centering
    \begin{tikzcd}
        U \arrow[r, "F"] & F(U) \arrow[d, "\rho_V^U"] \\
        V \arrow[u, hook] \arrow[r, "F"] & F(V)
    \end{tikzcd}
    \caption{Structural morphism of presheaves.}
\end{figure}
\end{definition}

\begin{remark}
    Intuitively, precheaves encode data from $C$ attached to open sets of $X$. The structural morphism $\rho_V^U: F(U) \rightarrow F(V)$ can be thought of as the restriction of the data enconded in $U$ to that concerning only the subset $V$.
\end{remark}

\begin{definition}
    If $F, G: \text{Op}(X) \rightarrow C$ are presheaves over a common topological space $X$, then a \emph{morphism} $m: F \rightarrow G$ is a collection of morphisms $m(U): F(U) \rightarrow G(U)$ from the category $C$ such that for every pair $V \subseteq U$ the following diagram commutes, where $\sigma_V^U$ are structural morphisms of the presheaf $G$.
\end{definition}

\begin{figure}[ht]
    \centering
    \begin{tikzcd}
        F(U) \arrow[d, "\rho_V^U"] \arrow[r, "m(U)"] & G(U) \arrow[d, "\sigma_V^U"] \\
        F(V) \arrow[r, "m(V)"] & G(V)
    \end{tikzcd}
\end{figure}

\begin{definition}
    \label{def:sheaf}
    A presheaf is called a \emph{sheaf} if it satisfies the following additional requirements: whenever $U = \bigcup_{\alpha \in A} U_\alpha$, where $U_\alpha \subseteq X$, then
    
    \begin{enumerate}
        \item for $f, g \in F(U)$, if $\forall \alpha \in A$
        \begin{equation*}
            \rho_{U_\alpha}^U f = \rho_{U_\alpha}^U g
        \end{equation*}
        then $f = g$.
        \item if $f_\gamma \in F(U_\gamma)$ for all $\gamma \in A$ are such that $\forall \alpha, \beta \in A$
        \begin{equation*}
            \rho_{U_\alpha \cap U_\beta}^{U_\alpha} f_\alpha
            = \rho_{U_\beta \cap U_\alpha}^{U_\beta} f_\beta
        \end{equation*}
        then there exists $f \in F(U)$ such that $\forall \gamma \in A$
        \begin{equation*}
            f_\gamma = \rho_{U_\gamma}^U f.
        \end{equation*}
    \end{enumerate}
\end{definition}

\begin{remark}
    Definition \ref{def:sheaf} implies that $f \in F(U)$ are determined by their values on an arbitrarily fine covering of $U$.
\end{remark}

\begin{definition}
    A \emph{fiber space} (or \emph{bundle}) over a base space $X$ is a continuous mapping $p: E \rightarrow X$.
\end{definition}

\begin{remark}
    A fiber space is a sheaf over $X$ with values in Set if for each $e \in E$ there exists a neighborhood $E'$ homeomorphic to $p(E') \times p^{-1}(p(e))$.
\end{remark}

\bibliographystyle{apalike}
\bibliography{notes}

\end{document}
