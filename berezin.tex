\documentclass{article}

\usepackage[a4paper, margin=2cm]{geometry}
\usepackage[utf8]{inputenc}
\usepackage[english]{babel}
\usepackage{amsmath}
\usepackage{amssymb}
\usepackage{amsthm}
\usepackage{microtype}
\usepackage{booktabs}
\usepackage{marginnote}
\usepackage{tikz}
\usetikzlibrary{cd}
\usepackage{wrapfig}
\usepackage[T1]{fontenc}
\usepackage{newpxtext,eulerpx}

\newtheorem{theorem}{Theorem}
\newtheorem{corollary}{Corollary}[theorem]
\theoremstyle{definition}
\newtheorem{definition}{Definition}
\newtheorem*{proposition}{Proposition}
\newtheorem*{remark}{Remark}
\newtheorem{example}{Example}

\title{Notes on Superanalysis}
\author{Rui Peixoto}
\date{}

\begin{document}
\maketitle

\section{Grassman Algebra}

\begin{definition}
    An \emph{algebra} is a vector space $\mathfrak{A}$ over a field $\mathbb{K}$ equiped with a product satisfying $\forall u, v, w \in \mathfrak{A}$ and $\forall \alpha \in \mathbb{K}$
    \begin{itemize}
        \item $u(v + w) = uv + uw$
        \item $(u + v)w = uw + vw$
        \item $\alpha (u v) = (\alpha u) v = u (\alpha v)$.
    \end{itemize}

    The algebra $\mathfrak{A}$ is said to be \emph{associative} if $\forall u, v, w \in \mathfrak{A}$
    \begin{equation*}
        (uv)w = u(vw)
    \end{equation*}
    and \emph{unitary} (or algebra with a unit) if $\exists e \in \mathfrak{A}$ such that $\forall u \in \mathfrak{A}$
    \begin{equation*}
        eu = ue = u.
    \end{equation*}
    Furthermore, $\mathfrak{A}$ is \emph{commutative} if $\forall u, v \in \mathfrak{A}$
    \begin{equation*}
        uv = vu
    \end{equation*}
    and $\dim \mathfrak{A} = n$ for $0 \leq n \leq \infty$ if $\mathfrak{A}$ has dimension $n$ as a vector space.
\end{definition}

\begin{remark}
    We will consider only the cases $\mathbb{K} \in \{\mathbb{R}, \mathbb{C}\}$.
\end{remark}

\begin{example}
    There exist important algebras without a unit. Important examples of such algebras are \emph{Lie algebras}.
\end{example}

\begin{proposition}
    Let $\mathfrak{A}$ be a non-trivial Lie algebra. Then $\mathfrak{A}$ does not have a unit.
\end{proposition}
\begin{proof}
    Let $x, y \in \mathfrak{A}$. Assume $\mathfrak{A}$ admits a unit $e \in \mathfrak{A}$, that is $\left[x, e \right] = x = \left[ e, x \right]$ for all $x \in \mathfrak{A}$. By the anticommutativity of the bracket this implies that
    \begin{equation*}
        e = \left[ e, e \right] = -\left[ e, e \right]
        \Longrightarrow e = 0
    \end{equation*}
    which contradicts the fact that $\mathfrak{A}$ is non-trivial.
\end{proof}

\begin{proposition}
    Let $M$ be a set and $\mathfrak{A}$ an algebra over a field $\mathbb{K}$. Then $\mathfrak{A}^M = \left\{ f: M \longrightarrow \mathfrak{A} \right\}$ is an algebra over $\mathbb{K}$. If $\mathfrak{A}$ is associative or commutative, then so is $\mathfrak{A}^M$.
\end{proposition}
\begin{proof}
    Let $f, g, h \in \mathfrak{A}^M$, and $\alpha \in \mathbb{K}$. Then $\forall x \in M$
    \begin{align*}
        f(x) \left[ g(x) + h(x) \right] = f(x) g(x) + f(x)h(x) 
        & \Longrightarrow f(g + h) = fg + fh \\
        \left[f(x) + g(x) \right] h(x)  = f(x) h(x) + g(x)h(x) 
        & \Longrightarrow (f + g)h = fh + gh \\
        \left[ \alpha f(x) \right] g(x) = f(x) \left[ \alpha g(x) \right]
         = \alpha f(x) g(x) & \Longrightarrow (\alpha f) g = f(\alpha g) = \alpha fg
    \end{align*}
    so $\mathfrak{A}^M$ is an algebra over $\mathbb{K}$. Associativity and commutativity inherited from $\mathfrak{A}$ are shown analogously.
\end{proof}

\begin{example}
    Consider a domain $U \subseteq \mathbb{R}^p$. Then $
        \mathcal{A}(U) = \left\{ f \in \mathbb{K}^U \ \middle| \ f \text{ is smooth} \right\} $
    is an algebra over $\mathbb{K}$. $\mathcal{A}(U)$ admits a subalgebra $\mathcal{P}(U) = \left\{ f \in \mathcal{K}^U \ \middle| \ f \text{ is polynomial} \right\}$.
\end{example}

Generators

\section{Supermanifolds}

\subsection{Sheaves}

\begin{definition}
    Let X be a topological space. We define Op$(X)$ to be the \emph{category of open sets of X}. The objects of Op$(X)$ are open subsets $U \subseteq X$ and morphisms are inclusions of open sets $V \subseteq U \subseteq X$ into eachother.

    \begin{figure}[h]
        \centering
        \begin{tikzcd}[column sep=small]
            V \arrow[rr, hook] \arrow[dr] &  & U \arrow[dl] \\
            & X &
        \end{tikzcd}
        \caption{Morphisms in Op$(X)$.}
    \end{figure}
\end{definition}

\begin{remark}
    Morphisms in Op$(X)$ induce a partial ordering of Op$(X)$.
\end{remark}

\begin{definition}
    Let $X$ be a topological space and $C$ a locally small category\footnote{Consider categories such as Set, Ab, Vect, or Ring.}. A \emph{presheaf} over $X$ with values in $C$ is a contravariant functor $F:\text{Op}(X) \rightarrow C$.

A presheaf includes
\begin{enumerate}
    \item an assignment of an object $F(U)$ from the category $C$ to each open set $V \subseteq X$,
    \item an assignment of a morphism $\rho_V^U: F(U) \rightarrow F(V)$ from the category $C$ to each pair $V \subseteq U$ in Op$(X)$,
    \item an identity morphism $\rho_U^U = \text{id}$,
    \item $\rho_W^V \rho_V^U= \rho_W^U$ for open sets $W \subseteq V \subseteq U \subseteq X$.
\end{enumerate}

\begin{figure}[h]
    \centering
    \begin{tikzcd}
        U \arrow[r, "F"] & F(U) \arrow[d, "\rho_V^U"] \\
        V \arrow[u, hook] \arrow[r, "F"] & F(V)
    \end{tikzcd}
    \caption{Structural morphism of presheaves.}
\end{figure}
\end{definition}

\begin{remark}
    Intuitively, precheaves encode data from $C$ attached to open sets of $X$. The structural morphism $\rho_V^U: F(U) \rightarrow F(V)$ can be thought of as the restriction of the data enconded in $U$ to that concerning only the subset $V$.
\end{remark}

\begin{definition}
    If $F, G: \text{Op}(X) \rightarrow C$ are presheaves over a common topological space $X$, then a \emph{morphism} $m: F \rightarrow G$ is a collection of morphisms $m(U): F(U) \rightarrow G(U)$ from the category $C$ such that for every pair $V \subseteq U$ the following diagram commutes, where $\sigma_V^U$ are structural morphisms of the presheaf $G$.
\end{definition}

\begin{figure}[h]
    \centering
    \begin{tikzcd}
        F(U) \arrow[d, "\rho_V^U"] \arrow[r, "m(U)"] & G(U) \arrow[d, "\sigma_V^U"] \\
        F(V) \arrow[r, "m(V)"] & G(V)
    \end{tikzcd}
\end{figure}

\begin{definition}
    \label{def:sheaf}
    A presheaf is called a \emph{sheaf} if it satisfies the following additional requirements: whenever $U = \bigcup_{\alpha \in A} U_\alpha$, where $U_\alpha \subseteq X$, then
    
    \begin{enumerate}
        \item for $f, g \in F(U)$, if $\forall \alpha \in A$
        \begin{equation*}
            \rho_{U_\alpha}^U f = \rho_{U_\alpha}^U g
        \end{equation*}
        then $f = g$.
        \item if $f_\gamma \in F(U_\gamma)$ for all $\gamma \in A$ are such that $\forall \alpha, \beta \in A$
        \begin{equation*}
            \rho_{U_\alpha \cap U_\beta}^{U_\alpha} f_\alpha
            = \rho_{U_\beta \cap U_\alpha}^{U_\beta} f_\beta
        \end{equation*}
        then there exists $f \in F(U)$ such that $\forall \gamma \in A$
        \begin{equation*}
            f_\gamma = \rho_{U_\gamma}^U f.
        \end{equation*}
    \end{enumerate}
\end{definition}

\begin{remark}
    Definition \ref{def:sheaf} implies that $f \in F(U)$ are determined by their values on an arbitrarily fine covering of $U$.
\end{remark}

\begin{definition}
    A \emph{fiber space} (or \emph{bundle}) over a base space $X$ is a continuous mapping $p: E \rightarrow X$.
\end{definition}

\begin{remark}
    A fiber space is a sheaf over $X$ with values in Set if for each $e \in E$ there exists a neighborhood $E'$ homeomorphic to $p(E') \times p^{-1}(p(e))$.
\end{remark}

\end{document}
